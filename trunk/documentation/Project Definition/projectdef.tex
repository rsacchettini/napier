\documentclass[a4paper,12pt]{article}
\usepackage{fancyhdr}
\usepackage{graphicx}
\usepackage[utf8]{inputenc}


\title{Project Definition  - Estate Agency}
\author{Group Project 3 - Group 2}
\date{March 4 2008}

\begin{document}
\maketitle
\newpage
\tableofcontents
\newpage


 

\section{Introduction}
The Napier Estate Agency is a firm which deals in the selling of property within the UK. At the moment all sales are made either in person at the branch, or by telephone. The firm has both buyers and sellers. Sellers list their properties with the agency who then advertise them on the front window of the branch. Buyers can then see these properties and enter the branch to find out more information or arrange to view the property.

The client feels that by advertising the properties only on the window of the branch, they are limiting the number of buyers attracted. They wish to reach out to more potential buyers, and make finding out information and arranging visits to properties easier for them. The client also wishes to allow sellers to list properties themselves. The group has decided that in order to achieve these goals an online estate agency site should be created.

The website needs to have several functions. It must allow anyone who wants to buy or sell a property to register on the website and get an account. With this account, the user will be able to add properties, browse and search all the properties offered, find out more detailed information about each property, and arrange visits to properties.

The system will also be used by employees of the estate agency. It must allow the estate agents to view all the properties. They will be able to verify properties added by users before they go online, they will then be able to edit and remove a property at any time. The estate agent should be able to see a list of potential buyers for each property and have some method to contact each one.

The website must be consistent throughout each page. As the majority of people using the site are likely to not be technical people the site has to be easy to navigate and intuitive. It is important that searching the properties is easy to do and effective.

The website will be created using groovy java. This is a relatively new programming language based on java. Its syntax is similar to that of java, but simpler. There are far fewer rules which will make coding the site relatively easy.

 

\section{Objectives}

The objectives of the project are to deliver a website that will:

\begin{itemize}
\item Allow a seller or a potential buyer to register an account and maintain their details.
\item Allow sellers to add a property to the site, and edit or remove them.
\item Allow buyers to browse and search properties.
\item Allow buyers to add and remove properties to a favourites list.
\item Allow buyers to find out information and arrange visits to properties.
\item Allow the estate agent to verify added properties and edit or remove them.
\item Allow the estate agent to update any information on the website.
\item Allow the estate agent to see details of properties and find out who is interested in them.
\end{itemize}

 

\section{Scope}

 

In this project we will analyse, design, build and test a fully working website. The end product will be a fully implemented website including all the requirements listed in the requirements specification.

 

\section{Business Case}

 

The company feel that they are not attracting as many potential buyers through advertising the properties on their window as they could do if they advertised online. They feel that it in order to attract more customers to the estate agency a website is necessary.

 

\section{Benefits}

 

The benefits that this website will bring to the estate agency are:

 
\begin{itemize}
\item More advertising space. More properties can be listed on a website than in a shop window.
\item Increased reach and reputation. People from all around the UK will be able to browse the properties rather than just people who live locally to the branch. People trust companies they know about, the website will help people to get to know the estate agency.
\item Easier for the customer. They can do everything from the comfort of their home.
\item Advantage over competition that do not have a website as great deal of potential customers search for products and services online.
\item The services will be available to anyone 24 hours a day, making it convenient for the customers.
\item The customers and company will know more about each other which increases market confidence.
\end{itemize}

\section{Risks}


\begin{center}
  \begin{tabular}{| p{3cm} | p{3cm} | p{1.8cm} | p{1.8cm} | p{3.2cm}|}
    \hline
    Risk & Risk Description & Probability & Effect & Mitigation (Assessment)\\ \hline
    Time Constraint & This will be a big hurdle as we only have a limited time to complete the task & High & Disaster & Plan well so we know how long each stage will take, and stick to the plan\\ \hline
    Lack of skills in groovy & We are using a technology no one in the team is familiar with & High & Tolerable & Change to use a language we are more familiar with\\ \hline
    Personal Illness & We are a small team so if someone is ill it is a problem & Medium & Minimal & Someone else will take over the task until the ill person is well\\
	\hline
  \end{tabular}
\end{center}


\section{Project Organisation}
\begin{itemize}
\item Project Organiser: Romain Bossut
\item Team Member 1: Romain Sacchettini
\item Team Member 2: Sofyane Khedim
\item Team Member 3: Nicky Ward
\item Team Member 4: Annie Cheung 
\end{itemize}
\end{document}
